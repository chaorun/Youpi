%ADASS_PROCEEDINGS_FORM%%%%%%%%%%%%%%%%%%%%%%%%%%%%%%%%%
%
% SAMPLE2.TEX -- ADASS XVII (2007)-- ADASS Conference Proceedings sample
% paper with complicated markup. Based on ADASS XI (01) version.
%
% This is a comprehensive example, meaning that we have made use of each
% of the capabilities of the LaTeX + the ASPCONF macro package that we think
% you may need to use. 
%
% Much of the input will be enclosed by braces (i.e., { }).  The
% percent sign, "%", denotes the start of a comment; text after it
% will be ignored by LaTeX.  You might also notice in some of the
% examples below the use of "\ " after a period; this prevents LaTeX
% from interpreting the period as the end of a sentence and putting
% extra space after it.   
% 
% You should check your paper by processing it with LaTeX.  For
% details about how to run LaTeX as well as how to print out the User
% Guide, consult the README file.  
%
%%%%%%%%%%%%%%%%%%%%%%%%%%%%%%%%%%%%%%%%%%%%%%%%%%
% 
\documentclass[11pt,twoside]{article}  % Leave intact
\usepackage{asp2006}
\usepackage{adassconf}

\begin{document}   % Leave intact

%-----------------------------------------------------------------------
%			    Paper ID Code
%-----------------------------------------------------------------------
% Enter the proper paper identification code.  The ID code for your paper 
% is the session number associated with your presentation as published 
% in the official conference proceedings.  You can find this number by 
% locating your abstract in the printed proceedings that you received 
% at the meeting, or on-line at the conference web site.
%
% This identifier will not appear in your paper; however, it allows different
% papers in the proceedings to cross-reference each other.  Note that
% you should only have one \paperID, and it should not include a
% trailing period.

\paperID{F4}

%-----------------------------------------------------------------------
%		            Paper Title 
%-----------------------------------------------------------------------
% Enter the title of the paper.
%
% EXAMPLE: \title{A Breakthrough in Astronomical Software Development}

\title{Youpi, a Web-based Astronomical Image Processing Pipeline}
       
%-----------------------------------------------------------------------
%          Short Title & Author list for page headers
%-----------------------------------------------------------------------
% Please supply the author list and the title (abbreviated if necessary) as 
% arguments to \markboth.
%
% The author last names for the page header must appear in one of 
% these formats:
%
% EXAMPLES:
%     LASTNAME
%     LASTNAME1 and LASTNAME2
%     LASTNAME1, LASTNAME2, and LASTNAME3
%     LASTNAME et al.
%
% Use the "et al." form in the case of four or more authors.
%
% If the title is too long to fit in the header, shorten it: 
%
% EXAMPLE: change
%    Rapid Development for Distributed Computing, with Implications for the Virtual Observatory
% to:
%    Rapid Development for Distributed Computing

%\markboth{Djorgovski, King, and Biemesderfer}{Collapsed Cores in Globular Clusters}
\markboth{Monnerville and Semah}{Youpi, a Web-based Astronomical Image Processing Pipeline}

%-----------------------------------------------------------------------
%		          Authors of Paper
%-----------------------------------------------------------------------
% Enter the authors followed by their affiliations.  The \author and
% \affil commands may appear multiple times as necessary.  List each
% author by giving the first name or initials first followed by the
% last name. Do not include street addresses and postal codes, but 
% do include the country name or abbreviation. 
%
% If the list of authors is lengthy and there are several institutional 
% affiliations, you can save space by using the \altaffilmark and \altaffiltext 
% commands in place of the \affil command.

%\author{S.\ Djorgovski\altaffilmark{1,2}, Ivan R.\ King}
%\affil{Astronomy Department, University of California, Berkeley, CA, USA}

\author{M.\ Monnerville}
\affil{Terapix, Institut d'Astrophysique de Paris, University of Pierre et Marie Curie, UMR 7095, Paris, F-75014, France}

\author{G.\ Semah}
\affil{Terapix, Institut d'Astrophysique de Paris, CNRS, UMR 7095, Paris, F-75014, France}

% Notice that some of these authors have alternate affiliations, which
% are identified by the \altaffilmark after each name.  The actual alternate
% affiliation information is typeset in footnotes at the bottom of the
% first page, and the text itself is specified in \altaffiltext commands.
% There is a separate \altaffiltext for each alternate affiliation
% indicated above.

%\altaffiltext{1}{Visiting Astronomer, Cerro Tololo Inter-American Observatory. 
%CTIO is operated by AURA, Inc.\ under cooperative agreement with the National
%Science Foundation} 
%\altaffiltext{2}{Society of Fellows, Harvard University} 
%\altaffiltext{3}{Patron, Alonso's Bar and Grill}

%-----------------------------------------------------------------------
%			 Contact Information
%-----------------------------------------------------------------------
% This information will not appear in the paper but will be used by
% the editors in case you need to be contacted concerning your
% submission.  Enter your name as the contact along with your email
% address.

\contact{Mathias Monnerville}
\email{monnerville@iap.fr}

%-----------------------------------------------------------------------
%		      Author Index Specification
%-----------------------------------------------------------------------
% Specify how each author name should appear in the author index.  The 
% \paindex{ } should be used to indicate the primary author, and the
% \aindex for all other co-authors.  You MUST use the following syntax: 
%
%    \aindex{LASTNAME, F.~M.}
% 
% where F is the first initial and M is the second initial (if used). Please 
% ensure that there are no extraneous spaces anywhere within the command 
% argument. This guarantees that authors that appear in multiple papers
% will appear only once in the author index. Authors must be listed in the order
% of the \paindex and \aindex commmands.

%\paindex{Djorgovski, S.}
%\aindex{King, I.~R.}
%\aindex{Biemesderfer, C.~D.}

\paindex{Monnerville, M.}
\aindex{Semah, G.}

%-----------------------------------------------------------------------
%			Subject Index keywords
%-----------------------------------------------------------------------
% Enter up to 6 keywords that are relevant to the topic of your paper.  These 
% will NOT be printed as part of your paper; however, they will guide the creation 
% of the subject index for the proceedings.  Please use entries from the
% standard list where possible, which can be found in the index for the 
% ADASS XVI proceedings. Separate topics from sub-topics with an exclamation 
% point (!). 

%\keywords{astronomy!globular clusters}
\keywords{astronomy!pipeline clusters}

% We reset the footnote counter for the hyperlink since it does not
% appear to recognize the previous 3 footnotes generated from the
% altaffilmarks.  

\setcounter{footnote}{3}

%-----------------------------------------------------------------------
%			       Abstract
%-----------------------------------------------------------------------
% Type abstract in the space below.  Consult the User Guide and Latex
% Information file for a list of supported macros (e.g. for typesetting 
% special symbols). Do not leave a blank line between \begin{abstract} 
% and the start of your text.

%\begin{abstract}          % Leave intact
%This is a preliminary report on surface photometry of the major
%fraction of known globular clusters, to see which of them show the
%signs of a collapsed core.  We also show off the results of some
%recreational mathematics.
%\end{abstract}

\begin{abstract} 
Youpi stands for ``YOUpi is your processing PIpeline''. It is a modern, 
easy to use web application providing high level functionalities to perform 
data reduction on scientific FITS images. Built on top of various open source 
reduction tools released to the community by Terapix, Youpi can help organize 
your data, manage your processing jobs on a computer cluster in real time 
(using Condor) and facilitate teamwork by allowing fine-grain sharing of 
results and data.
\end{abstract}

%-----------------------------------------------------------------------
%			      Main Body
%-----------------------------------------------------------------------
% Place the text for the main body of the paper here.  You should use
% the \section command to label the various sections; use of
% \subsection is optional.  Significant words in section titles should
% be capitalized.  Sections and subsections will be numbered
% automatically. 

\section{Introduction}

Many data reduction pipeline softwares are available today. Youpi is an attempt to
build a rather generic and versatile web-based pipeline suitable for both large 
surveys and smaller PI data processing. It currently supports working with images 
from the Megacam, WIRCam and Ultra-VISTA instruments, and we plan to add support 
for more in the future. 

Youpi is designed to help in data management by keeping track of all data locations 
and parameters involved during the pipeline's life cycle, from the moment FITS images 
are ingested into the database to the moment final processing results are produced.

This pipeline tends to be quite versatile since it can be used on a single host -- a 
desktop computer or even a laptop -- or on larger installations such as computer 
clusters. The project is actively maintained at Terapix, published under an open 
source license and has first been released to the community on September 2009.

\section{Handling FITS Data Ingestion}

Ingesting images is probably the main entry point for firstime users in using Youpi.
Indeed, Youpi provides a way to keep track of your images so you can use them for later 
processing. By selecting -- with a graphical tool (Fig. \ref{fig:pathsel}) -- 
one or several data paths to your FITS images (available either locally or on the 
network by using network shares such as NFS), you will instruct Youpi to read their 
headers and add relevant information to the database, then you will get a detailled 
email report at the end of ingestion. Now that images are ingested once and for all, 
they are available for later processing and tools are provided (see section \ref{sec:ims}) 
to select them by making criteria-based image selections.

\begin{figure}[h]
\epsscale{0.20}
\plotone{ingestion.eps}
\caption{Using the data path selector during image ingestion. Three data paths have 
been selected. Everyone of them will be queued for ingestion on the Condor Cluster.}\label{fig:pathsel}
\end{figure}

\section{Data Processing}

Youpi is a high-level tool acting as a wrapper for freely available, low-level software 
packages released to the astronomer community by Terapix. Through the use of these packages, 
we currently support image quality assessment using the \emph{QualityFITS} software, 
computing astrometric and photometric solutions using \emph{SCAMP}, resampling and co-adding 
images using \emph{SWarp} and extracting sources and building source catalogues using 
\emph{SExtractor}.

All of these processings are internally available as plugins. Some of them can be disabled 
in order to meet your needs. This plugin architecture is very modular and comprehensive 
documentation is available for writting custom plugins that would extend Youpi's default 
processing capabilities.

Since sharing data and results with other people such as PIs or teammates will certainly 
be a necessity at some point, Youpi ensure that only authorized people gain access to data 
you want to expose by simulating a Unix-like file permission system. Most of the data sets 
and results produced can be given permissions for the owner, a group and the rest of the 
world. This way you can define access policy for members of a group and decide whether non 
members can access the data (Fig. \ref{fig:permissions}).

\begin{figure}[h!]
\epsscale{0.40}
\plotone{permissions.eps}
\caption{A QualityFITS result set has been given read-write permission for the owner, read-only 
access for all members of group "Terapix" an no access at all for the others.}\label{fig:permissions}
\end{figure}

\section{Data Management}

Being able to assist the astronomer in data management and data sharing has been an important 
goal since the beginning of the project. As such, Youpi comes with many widgets and tools that 
make selecting and organizing data or dealing with configuration files and results a breeze.

\subsection{The Data Path Selector}

Whichever processing plugin you have selected, an embedded path selector is available so you 
can select and save data paths in real time from within your Internet browser (Fig. \ref{fig:browser}).
For example, if you want to compute astrometric and photometric solutions for a bunch of images 
using the \emph{SCAMP} plugin, you will have to provide a path to external \emph{.ahead} files.
The path selector widget will let you browse local -- and remote, if you are using network mounts -- 
filesystems, then you will be able to save those directories for later reuse.

\begin{figure}[h]
\epsscale{0.70}
\plotone{browser.eps}
\caption{The path selector widget (on the left) is used to select a directory holding \emph{.ahead} 
files required for a \emph{SCAMP} processing.}\label{fig:browser}
\end{figure}

\subsection{The Image Selector}\label{sec:ims}

The image selector (Fig. \ref{fig:ims}) is a tool dedicated to build selections of previously 
ingested images.  Using it is a required step since Youpi processings need FITS images as input. 
Several facilities are provided in order to make selecting images as painless as possible. 
Multiple search criteria can be combined at a time. Ingested images can be searched by \emph{RUNID}, 
\emph{channel}, \emph{grade} -- since Youpi provides an integrated interface for image 
quality assessment, \emph{image name} (using regular expressions), \emph{ingestion ID}, 
\emph{instrument name}, \emph{object name}, \emph{tag} and \emph{saved selection}.

Once a selection of images has been made, basic editing commands can be used to save the 
current selection, and saved selections can be deleted or merged with any other. In order 
to increase the image selector's flexibility in building selections of images, options have 
been added to allow uploading a plain text file describing the selection (one image name 
and optional checksum per line) or batch importing all text files within a directory. 
These two features allow creating saved selections on-the-fly rather than doing it manually.
This can be interesting if you are working on a large survey involving lots of images because
you can decide to prepare all your selections of images first before starting processing them.

\begin{figure}[h]
\epsscale{0.70}
\plotone{ims.eps}
\caption{A selection of 1450 images is built by retreiving all images from the 
\emph{CFHTLS-T0006-W3\_Scamp} saved selection.}\label{fig:ims}
\end{figure}

\subsection{Organize Data With Tags}
\subsection{The Processing Cart}

\section{Processing Results}
toto
\section{Cluster Computing With Condor}
Aaaaa
\subsection{Embedded Condor Policies}
BBBBBB
\subsection{Live Monitoring of Processings}
CCCCC

\acknowledgments

We are grateful to V.\ Barger, T.\ Han, and R.~J.~N.\ Phillips for doing
the math in section \ref{hairymath} of this paper.

%-----------------------------------------------------------------------
%			      References
%-----------------------------------------------------------------------
% List your references below within the reference environment
% (i.e. between the \begin{references} and \end{references} tags).
% Each new reference should begin with a \reference command which sets
% up the proper indentation.  
%    NOTE: all citations in the text _must_ have a corresponding entry in 
%    the reference list, and all references must be cited in the text.
%
% Observe the following order when listing bibliographical 
% information for each reference:  author name(s), publication 
% year, journal name, volume, and page number for articles. 
% URLs to the reference may be given either in-line, or as a footnote. 
% Note that many journal names are available as macros; see
% the User Guide for a listing "macro-ized" journals. 
%
% You may find the following tricks to be helpful:
%
%   o  "\ " after a period prevents LaTeX from interpreting the period 
%      as an end of a sentence.
%   o  \adassxvi is a macro that expands to the full title, editor,
%      and publishing information for the ADASS XVI conference
%      proceedings.  Such macros are defined for ADASS conferences I
%      through the most recently published proceedings.
%   o  When referencing a paper in the current volume, use the
%      \adassxvii and \paperref macros.  The argument to \paperref is
%      the paper ID code for the paper you are referencing.  See the 
%      note in the "Paper ID Code" section above for details on how to 
%      determine the paper ID code for the paper you reference.  

\begin{references}

\reference Auri\`ere, M.\  1982, \aap, 109, 301
\reference Djorgovski, S., \& King, I.\ R.\  1984, \apj, 277, L49
\reference Hagiwara, K., \& Zeppenfeld, D.\  1986, Nucl.Phys., 274, 1
\reference H\'enon, M.\  1961, Ann.d'Ap., 24, 369
\reference King, I.~R.\  1966, \aj, 71, 276
\reference King, I.~R.\  1975, in Dynamics of Stellar Systems, ed.\ A.\ Hayli
    (Dordrecht: Reidel), 99
\reference King, I.\ R., Hedemann, E., Hodge, S\ M., \& White, R.~E.
    1968, \aj, 73, 456
\reference Kron, G.~E., Hewitt, A.~V., \& Wasserman, L.~H.\
    1984, \pasp, 96, 198
\reference Lynden-Bell, D., \& Wood, R.\  1968, \mnras, 138, 495
\reference Ortolani, S., Rosino, L., \& Sandage, A.\  1985, \aj, 90, 473
\reference Peterson, C.~J.\  1976, \aj, 81, 617
\reference Roberts, D.~A.\ 2008, \adassxvii, \paperref{P1.3}
\reference Spitzer, L.\  1985, in Dynamics of Star Clusters,
    ed.~J.~Goodman \& P.~Hut (Dordrecht: Reidel), 109
    
\end{references}

% Do not place any material after the references section

\end{document}  % Leave intact
